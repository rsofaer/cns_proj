
\documentclass{article}

\usepackage{graphicx}
\usepackage{amssymb}
\usepackage{lastpage}
\usepackage{epstopdf}
\usepackage{fancyhdr}

\newcommand{\cAuthor}{Raphael Sofaer}
\newcommand{\cTitle}{Noise in an 8 Neuron Clione CPG simulation}
\pagestyle{fancy}
\lhead{\cAuthor}                                                 %
\rhead{\cTitle}  %
\lfoot{\lastxmark}                                                      %
\cfoot{}                                                                %
\rfoot{Page\ \thepage\ of\ \pageref{LastPage}}                          %
\renewcommand\headrulewidth{0.4pt}                                      %
\renewcommand\footrulewidth{0.4pt}        

\title{\cTitle}
\author{\cAuthor}
\begin{document}
\maketitle

\section*{Introduction}
In his book Spikes, Decisions and Actions (SDA), Wilson presents an 8 neuron
model of swimming in Clione based on the work of Satterlie and Spencer (1985).
The model is two connected instances (dorsal and ventral) of a 4-neuron system.
The two sides have an inhibitory connection to each other, and a spike on one side
generates an inhibitory pulse on the other side, then a post-inhibitory rebound
sufficient to trigger a spike, generating a stable out-of-phase pattern of spikes.
With no initial stimulus, the system remains in a stable rest state.

\scalebox{0.6}{\input{Normal.tex}}
\scalebox{0.6}{\input{Rest.tex}}

Below, we examine the effect of noise on this model, determining the 
level of noise at which the characteristic out of phase synchrony is no longer stable,
the level of noise at which the system is no longer stable at rest,
and examining the other regimes which emerge as the system becomes noisy.
We also look for some of the features of the Clione CPG observed in Satterlie's paper (1985).
He found phase-locked sub-threshold activity, as well as phase-locked spiking, and observed
alternating spikes and sub-threshold potentials as well.

\section*{Methods}
All simulations were run in GNU Octave, using a version of the Clione.m included
in SDA modified for flexibility, to include noise, and to use Forward Euler integration.
Noise was generated by first generating Gaussian white noise, then filtering it to 
correspond to a $1/f^x$ ``pink noise'' pattern.

\section*{Results}
\section*{Conclusions}
\end{document}
